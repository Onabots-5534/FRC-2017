\chapter{Software Configuration}

Programming and running the Onabots Robot requires six components. Each of these components requires the installation of software and configuration. Wherever possible, free open-source software is used.

\begin{enumerate}[label=$\Box$]

	\item Coding Computer. The Onabots use multiple computers running Debian GNU/Linux. Additional software includes OpenJDK 8, Eclipse with WPI plugins, Git, and ssh client.

	\item Driver Station. A Windows computer is required to control the robot as well as configure the OpenMesh Radio and the RoboRIO.

	\item OpenMesh Radio. The radio may be configured for wireless control at home but is reconfigured at tournaments to be restricted to the field wireless system.

	\item Raspberry PI. This computer runs the Raspian OS, a Debian OS for the Raspberry PI. The only other software this computer uses is Python3 with OpenCV for vision processing.

	\item RoboRIO. This requires two pieces of software to be installed: firmware and Java 8. If the firmware is updated, Java 8 will need to be reinstalled. As of Jan 2018, the firmware installation also installs Java automatically.

	\item Server. Computer running Debian GNU/Linux. Provides a web server and a version control system using git.
	
	\item Github
	
	\item Slack

\end{enumerate}


	\subsection*{Networking}
	Check the FRC web site to ensure that these IP addresses do not conflict with the Field Management System, DHCP, or other addresses. Also confirm that the netmask are valid.
	\vspace*{3mm}

	{\renewcommand{\arraystretch}{1.5}
	\begin{tabular}{ @{} l l l p{3in} }
	\textbf{Component} & \textbf{Address} & \textbf{Netmask} & \textbf{Notes} \\
	\midrule
	OpenMesh Radio & 10.55.34.1 & n/a & Set automatically by radio configuration utility \\
	RoboRIO        & 10.55.34.2 & 255.255.255.0 \\
	DriverStation  & 10.55.34.5 & 255.0.0.0     & Note the different netmask for the driver station \\
	Raspberry PI   & 10.55.34.6 & 255.255.255.0 & Address does not matter unless not using networktables \\
	\end{tabular}
	}




\newpage\section*{Coding Computers}

\begin{itemize}
\item OpenJDK 8
\item Eclipse
\item Git
\end{itemize}


\section*{Driver Station}
\begin{itemize}
\item Uninstalll previous National Instruments software: Control Panel -> Programs and Features ...
\item FRC 2018 Update Suite - link is on FRC website, in 2018 we used http://www.ni.com/download/first-robotics-software-2017/7183/en/
\item Credentials are ssteensma@oacsd.com / ninja5534
\item Serial number is on team package: for 2018 ours was M83X15194
\item Download is around 800 MB from First
\item Installation (full) takes from 10--20 minutes
\item I skipped registration since we do not use LabView Vision Development. (Not sure if this is correct).
\end{itemize}


\section*{GitHub}
\begin{itemize}
\item Team github credentials are ???
\end{itemize}


\section*{OpenMesh Radio}
\begin{itemize}
\item Uninstall FRC Radio Configuration Utility
\item Download FRC Radio Configuration Utility - link on FRC website. Search for FRC Radio Configuration (YEAR). Version on 2018-01-08 is 18.1.0
\item Do not use the IL version - that is for team in Israel


\item Firmware
\item Configure
\end{itemize}


\section*{Raspberry PI}
\begin{itemize}
\item Raspian OS
\item Python3
\item Other stuff
\item OpenCV
\end{itemize}


\section*{RoboRIO}
\begin{itemize}
\item Firmware
\item Oracle Java
\end{itemize}


\section*{Server}
\begin{itemize}
\item apache2
\item git
\item ssh
\end{itemize}



Also recommend using a combination of Git server and GitHub for public code sharing.
%
I recommend keeping all documentation, vision code, and robot code in the same repository.



\newpage\section*{OpenMesh Radio}

\begin{itemize}
\item Label radio with team number and year with piece of masking tape. This is needed at tournaments when the field judges configure the radio. Keep a radio configured for home use so that it does not have to be reconfigured. This also allows us to use the robot wirelessly during practice. Indicate whether this is the tournament or home radio.

\item Configure the radio. This may need to be check later to see if there is a more up-to-date version. Note that `IL' represents a version used in Israel.

\item Radio is powered by the VRM 12V/2A port. Strip 5/16 inn from wires.

\item For power, the center of the barrel is positive. Red should be connected to the center. Black is for the outside.
\end{itemize}


\subsection*{Firmware}

Must be performed using a Windows computer.
 
Note that the installation does not begin with the radio connected to the computer.

\begin{enumerate}
\item Disable WiFi on Control Panel. It is not enough to just turn off WiFi or use airplane mode.

\item Open FRC Radio Configuration Utility from desktop

\item Select Ethernet as network interface

\item Enter team number (e.g., 5534)

\item Make sure the radio is on OpenMesh

\item Press `Load Firmware'

\item Follow on-screen directions pertaining to power cycling

\item After the firmware is loaded, wait for about 2 minutes until the power light has stopped flashing for about a minute.

\item Press `Configure'

\item Done
\end{enumerate}

